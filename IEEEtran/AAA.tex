\documentclass[conference]{IEEEtran}
\IEEEoverridecommandlockouts
% The preceding line is only needed to identify funding in the first footnote. If that is unneeded, please comment it out.

\ifCLASSOPTIONcompsoc
  \usepackage[nocompress]{cite}
\else
  % normal IEEE
  \usepackage{cite}
\fi

\usepackage{caption}
\usepackage{graphicx, subfigure}
\usepackage{algorithm}
\usepackage[noend]{algorithmic}
\usepackage{textcomp}
\usepackage{multirow}
\usepackage{enumitem}
\usepackage{amsmath,amssymb,amsfonts}
\def\BibTeX{{\rm B\kern-.05em{\sc i\kern-.025em b}\kern-.08em
    T\kern-.1667em\lower.7ex\hbox{E}\kern-.125emX}}
\renewcommand{\algorithmicrequire}{ \textbf{Input:}} %Use Input in the format of Algorithm
\renewcommand{\algorithmicensure}{ \textbf{Output:}} %UseOutput in the format of Algorithm
%\renewcommand\thesection{\arabic{section}} 
%\renewcommand\thesubsection{\thesection.\Alph{subsection}} 

\begin{document}

\title{ Agile Adaptive Flow-Level Co-Sampling Strategy for SDN}

\author{He Cai$^{1}$, Jun Deng$^{1}$, Xiaofei Wang$^{1}$
\\
${^1}$Tianjin Key Laboratory of Advanced Networking, %School of Computer Science and Technology,\\
Tianjin University, Tianjin, China.
}


\maketitle

% As a general rule, do not put math, special symbols or citations
% in the abstract
\begin{abstract}
With the proliferation of Internet applications and the explosion of traffic, Fine-gaine's flow-level information acquisition provides basic support for network management, TE, security analysis, and Qos.In traffic sampling, due to the scale of the network, the analysis capcity of the Collector,and the highly dynamic network, there are a large number of flows cannot be captured. In this paper, we focus on maximizing sampling accuracy and sample effective ratio,proposing a collaborative sampling model based on influence, which combines the three dimensional influences of the nodes to quantify the value of the nodes: the direct influence of the current active flow of nodes, the topological influence of the nodes, the historical influence of the nodes; And through the overlapping relationship of active flows between nodes, the cooperative sampling relationship of nodes at sampling timing is constructed. Based on this model, we propose the Agile Flow-Level Co-sampling strategy(AFCS), which solves the approximate optimal solution of CSBI through three steps: sampling point selection, sampling time allocation, and collaborative strategy optimization. We evaluated our strategy through a real large-scale topology, and the results show that it can effectively improve the sampling accuracy, especially in the mice-flow accuracy, and effectively improve the sampling efficiency.
\end{abstract}
%from the three dimensions of sampling node selection, time allocation and collaboration between nodes.
\IEEEpeerreviewmaketitle

\section{Introduction}

With the data traffic and network scale rapidly increaing, there exists huge demand for scalable network management. Flow-based measuresment is crucial to the network management. Accutate,timely and efficient flow-level statistics collection can provide assistance for security analysis,traffic engineering and intelligent routing. 

To solve the above problems, a agile, adaptive and cooperative sampling mechanism which can be applied to large-scale data center network is urgently needed. 

In this paper, we concentrate on how to address the above limitation and to achieve accurate and timely flow detection in the maximal value without modification to existing physical facility and extra burden to data center networks


\begin{figure}[!hhhhhhhhhht]
\centering
\includegraphics[width=8.5cm]{images/png_architecture.png}
\caption{Co-Sampling Architecture}
\label{Architecture}
\end{figure}

\section{Related Work}
Our main contributions are summarized as follows:
\begin{itemize}[leftmargin=*]
\setlength{\parindent}{0pt}

\item In the context of Flow-level traffic sampling, we first proposed the concept of cooperative sampling, and explained the effect of timing cooperation between nodes on sampling accuracy and sampling efficiency.
\item For the first time, we modeled the problem of maximizing flow sampling accuracy. It constructs the cooperative relationship of the nodes through the overlapping relationship between the nodes, and quantifies the value of the nodes in the sampling period through the three dimensions of the node's current value, topological value and historical value.
\item Based on this model, we propose an AFCS strategy for solving the approximate optimal solution of the model. The strategy consists of three algorithms: \textbf{Iterative comprehensive influence-based $Top-K$ node selection based on flow privatization}, \textbf{Non-starved influence change-aware priority polling slot allocation}, \textbf{Co-Sampling slots ordering based on effectiveness maximizing greed}.And, We evaluated it through a real large-scale network topology.
\end{itemize}

\section{SYSTEM DESCRIPTION AND PROBLEM STATEMENT }

Figure \ref{Architecture} briefly demonstrates the process of node cooperative sampling in AFCS.Under the constraints that C packets/T is the maximum Collector processing capability and no more than K sampling nodes, at the end of each round of sampling period, before the start of a new round of sampling period T,
In order to satisfy the maximum sampling accuracy in T, each node is assigned a different number of sampling time slots, and these sampling nodes are arranged in a certain order in the sampling timing, and all the nodes work together to maximize the sampling accuracy of the flow.

In a large-scale network, under the constraints of given $C$ and no more than $K$ nodes, the challenge to achieve maximum flow-level sampling accuracy is how to select $K$ sampling nodes and how to allocate sampling time for these sampling points in a new round of sampling period T. For nodes selection and time allocation, not only the flow information known at the current time should be considered, but also the possibility that new flows arrive at each node in the T period after the sampling strategy is issued.In addition, another important reason that affects the sampling accuracy is the repetition rate of the sample packets, which represents the ratio of the number of repeated packets to the number of all sampled packets during the sampling process.Repeated packets not only occupy a limited amount of sampling resources, but also reduce the effectiveness of sampling, and also reduce the sampling accuracy.The higher the repetition rate, the more collector processing resources are wasted, and the space for improved sampling accuracy will be reduced or even reduced.The space with improved sampling accuracy will be compressed and even reduce the sampling accuracy.

The repetitive packet is generated because the flows in the network may pass through multiple nodes at the same time. At the same time, more than one of these nodes is sampled, and a duplicate packet will be generated.That is to say, the overlapping relationship of the flows between the nodes and the arrangement of the nodes at the sampling time determine the repetition rate.Reducing the unnecessary repetition rate provides more room for improvement in sampling accuracy.From this perspective, in a sampling model that maximizes the accuracy of flow sampling, it should include the value of the arrangement of each node in the sampling time,the overlapping relationship of the flows between the nodes and the arrangement of the nodes in the sampling time dimension are included in the process of sampling node selection and time allocation.Sometimes the repetition rate is impossible to avoid, and it is necessary if the value brought by the repetition rate can maximize the sampling accuracy.

\subsection{PROBLEM Model}
Maximizing the accuracy of stream sampling is like maximizing area coverage, as shown in Figure\ref{fig_1_model}.The red solid line area represents all possible flows in the new round period T, including the currently known flows and possibly within T.Each gray area and red dashed area are respectively the currently known flows passing through the node and the new flows that may arrive at the node within T.When maximizing the area coverage, the area of the gray area of the node and the area of the red dotted area should be fully considered, and the overlap of the areas between the nodes should also be considered.As mentioned above, for the value generated by the node, we should not only consider the value brought by its current flows(\textbf{Direct Value}), but also the value brought by the new arrival flows in T(\textbf{Potential Value}).
\begin{figure}[!!!!!!!!!!!!!!hhhhhhhhhht]
\centering

\includegraphics[width=7cm]{images/area_coverage.png}
\label{fig_1_area}

\caption{Area coverage problem}
\label{fig_1_model}
\end{figure}
In order to establish a model that maximizes the accuracy of flow sampling, it is necessary to quantify the value of each node and the cost generated by the node, and combine the overlapping relationship of the flows between nodes and the cooperative relationship of the nodes in the time dimension.For the direct value of the node $R_i$, $F^c_i$ is the currently known set of flows passing through it,and any packet of $f_k$ is captured by $R_i$, which means that $f_k$ was successfully captured by $R_i$.Assuming that the arrival of packets of a flow $f_k$ obeys the poisson distribution with ${\lambda_{k}}$,Then, during the sampling time t, $R_i$ captures the expected number of current flows is $\sum_{f_k \in F^c_i}{P\{N_p^k(t) > 0\}}$.We can easily quantify the direct value that a node brings.But for the potential value of a node, we can't know the number of flows that each node arrives in T in the future. Even by analyzing the arrival distribution model of the flow, it is impossible to know the overlapping relationship of these unknown flows between the nodes.Therefore, the potential value of a node requires an independent way of quantification (independent of other nodes).

We propose a quantitative approach based on the influence and unify direct value and potential value.Let $t$ is the unit time, $t<=T$. And let $L = T/t$ represents the number of time slots.The direct influence of any node $R_i$ within $t$ can be quantified as (\ref{model_di}).It is described as the expected number of known flows captured by the node within $t$ divided by the total number of all known flows.The quantization method is based on the extended form of the flow betweenness centrality[2], which reflects the influence of the node in current known flow.
\begin{equation}
D_i= \sum\limits_{{{f}_{k}}\in {{F}^{c}}}{P\left\{ N_{p}^{k}\left( t \right)>0 \right\}} /|F^c|
\label{model_di}
\end{equation}
The potential value of a node, we evaluate from two dimensions: the influence of the node location in the network topology [3], and the proportion of the number of node historical flows.We use standard intermediaries to measure the influence of nodes in the network topology[2], which we call the topological influence of nodes, denoted as $S_i$.It represents the importance of the location of the node in the network topology;The proportion of the number of historical flows of the node reflects the activity of the node throughout the network lifecycle, denoted as $H_i$, $H_i=TF_i/TF$,where $TF_i$ represents the number of flows that $R_i$ has passed so far, and $TF$ is the total number of flows that have passed through the network so far.

Therefore, the potential value of the node in the next sampling period T can be quantified using $S_i$ and $H_i$.Within t, the comprehensive  value generated by the node can be expressed as: $\alpha \cdot D_i +  \beta \cdot \frac{S_i}{L} + \gamma \cdot \frac{H_i}{L}$. Where $\alpha + \beta + \gamma =1$ represents the weighted sum of the three influences.

The cost of a node is the total number of sample packets generated at a given time. We use $w_i$ to represent the total number of packets generated by $R_i$ within $t$. $w_i$ can be expressed as formula(\ref{wi}), where $v_i$ is the current packet rate (packets/T) of $R_i$, because there is subsequent unknown traffic coming, we use the ratio of the potential value of the node in $t$ to the direct value of the node in $t$ to infer their rate.
\begin{equation}
{{w}_{i}}= \frac{1}{L} \cdot [v_i + v_i \cdot \frac{\beta \cdot \frac{S_i}{L} + \gamma \cdot \frac{H_i}{L}}{\alpha \cdot D_i}]
\label{wi}
\end{equation}
We give the Influence-based maximization flow sampling accuracy collaborative sampling model in (\ref{1}),where ${{s}^{l}}($l=1,2...,L$)$ represents the different slots in T. (\ref{2}) (\ref{3}) describe the constraints that the model should satisfy. 

\begin{equation}
\begin{split}
%\begin{gather}
\max \sum\limits_{i}^{n}{(\alpha \cdot {\delta ({{D}_{i}},\left| \widetilde{{{S}_{i}}} \right|)}+\frac{\left| \widetilde{{{S}_{i}}} \right|}{L\;}\cdot (\beta \cdot {{S}_{i}}+\gamma \cdot {{H}_{i}}))}  \\
-\frac{\alpha }{\left| {{F}^{c}} \right|}\cdot \sum\limits_{{{f}_{k}}\in {{F}^{c}}}{\sum\limits_{l=1}^{L\;}{\left( P\left\{ N_{p}^{k}\left( t \right)>0 \right\}\cdot \psi \left( {{f}_{k}},{{s}^{l}} \right) \right)}} \label{1}
%\end{gather}
\end{split}
\end{equation}

\begin{equation}
subject to: \sum\nolimits_i^n {f(\widetilde {{S_i}}) \le } K,f(\widetilde {{S_i}}) = \left\{ \begin{array}{l}
1,\left| {\widetilde {{S_i}}} \right| \ge 1\\
0,\left| {\widetilde {{S_i}}} \right| = 0
\end{array} \right.\label{2}
\end{equation}
\begin{equation}
\sum\nolimits_{i}^{n}{{{w}_{i}}\cdot \left| \widetilde{{{S}_{i}}} \right|\le C}\label{3}
\end{equation}
The goal of the model is to allocate the corresponding sampling time for each $R_i$ under the constraints of given $C$ and $K$, and to determine the sampling cooperation strategy among nodes, so as to maximize the influence of the system in the period $T$, and then maximize the flow-level sampling accuracy. $\widetilde{{{S}_{i}}}$ represents the set of slots allocated to $R_i$.The larger the size $|\widetilde{{{S}_{i}}}|$ of slot set, the greater the comprehensive influence of nodes.The relationship between the comprehensive influence of nodes and $|\widetilde{{{S}_{i}}}|$ is given in (\ref{4}).
\begin{equation}
\delta \left( {{v_i},\left| {\widetilde {{S_i}}} \right|} \right) = \left\{ \begin{array}{l}
{v_i} \cdot \left| {\widetilde {{S_i}}} \right|,{\rm{    }}{v_i} \cdot \left| {\widetilde {{S_i}}} \right| < \left| {F_i^c} \right|\\
\left| {F_i^c} \right|{\rm{   }}\quad,\ {\rm{    ELSE}}
\end{array} \right.\label{4}
\end{equation}
When $ D_i\cdot |\widetilde{{{S}_{i}}}|=|F^c_i|$, it means that nodes can capture all the flows currently detected.Then its direct value will not continue to increase with the increase of $|\widetilde{{{S}_{i}}}|$, but the potential value will continue to increase. In addition, when a flow is collected by different nodes in the same slot, the actual number of flow in the whole system does not increase. Therefore, the value generated by a single slot of a node is related to the number of slots the node already has, not independent.

In (\ref{1}), the first part represents the sum of the combined values of all nodes under their respective $\widetilde{S}_i$, and the subtraction part indicates the sum of the direct influences of the resampled flows at all the same time slots. $U = \left\{ {{R_i},{f_k} \in F_i^c \wedge {s^l} \in \widetilde {{S_i}}} \right\}$ is a set of routers which contain $f_k$ and sampling at $s^l$.  $\psi ({{f}_{k}},{{s}^{l}})$,tt calculates the number of times each flow is resampled at slot $s^l$. As mentioned above, the packet repetition rate is generated by more than one nodes sampling the same flows at the same time.Therefore, the model takes the overlapping relationship of flows between nodes and the arrangement of nodes in the sampling time dimension as one of the factors that influence the accuracy of flow sampling.  %%%%%
%%%%%%%%%%%%%%%%%%%%%%%%%%%%%%%%%%%%%%%%%%%%%%%%%%%%%

\begin{equation}
\psi \left( {{f_k},{s^l}} \right) = \left\{ \begin{array}{l}
\left| U \right| - 1,{\rm{    }}\left| U \right| \ge 1\\
{\rm{   }}0\quad \quad \; \; ,\, \left| U \right| = 0
\end{array} \right.\label{5}
\end{equation}

We explained the quantization process of the model and illustrated it as Figure\ref{fig_1_model}.The goal of the model is to allocate reasonable $\widetilde{{{S}_{i}}}$ sets for all $R_i$ under constraint conditions.As  figure\ref{fig_1_model} shows, when the system gets the optimal solution of $S ^ 1..S ^ n $, it not only reflects the optimal node selection, the optimal time allocation, but also reflects the optimal slot order arrangement, which is determined by the cooperation strategy between nodes. At the same time, each node actually avoid overlapping on the same slot as much as possible, so that the effective ratio of the system is improved.

 

\begin{figure}[!!!!!!!!!!!!!!hhhhhhhhhht]
\centering

\includegraphics[width=8.5cm]{images/slot_num_order.png}
\label{fig_1_slot}

\caption{Overview of model}
\label{fig_1_model}
\end{figure}



\section{Agile Flow-Level Co-sampling strategy(AFCS)}
Since the above model is complicated, we divide it into three sub-problems: K node selection, Slot allocation, and collaborative sampling optimization. The solution to each sub-question will be the input to the next sub-question. Through three steps, the approximate optimal solution of the total problem is obtained.In this section, we detail each sub-question and propose the corresponding algorithms.
\subsection{Iterative comprehensive influence-based $Top-K$ node selection based on flow privatization} 
In the selection of K sampling points, we only focus how to use the static properties of the nodes to quantify their comprehensive influence, which is independent of time allocation. In (\ref{1}), when the number of slots and the cooperative relationship between nodes (slots order) are not considered, the optimization formula can be expressed as (\ref{maxk}) and the constraint is (\ref{maxkc});The goal of the optimization model \ref{maxk} is to select K sampling nodes without considering the influence of time on the comprehensive influence of the nodes, so as to maximize the comprehensive influence of the system.Similar to the CIMCS model, the former part sums the comprehensive influence of the selected nodes, and the latter part subtracts the influence part of the repeated calculations.
\begin{equation}
\max [\sum_{i=1}^n (\alpha \cdot {D_{i}} + \beta \cdot {S_{i}} + \gamma \cdot {H_{i}}) \cdot \varphi{(i)} - \alpha \cdot\sum_{f_k \in F^c} \widehat{\psi}{(f_k)}]
\label{maxk}
\end{equation}
\begin{equation}
subject to:\sum_{i=1}^{n} \varphi(i) = K
\label{maxkc}
\end{equation}
In(\ref{maxk}), $D_i=|F^c_i|/|F^c|$,and $\varphi(i)$ is 0 means that $R_i$ is not selected, 1 means $R_i$ is selected;$\widehat{U} = \{R_i,f_k \in F^c_i \wedge \varphi(i) = 1\}$ indicates how many selected nodes cover $f_k$,so the(\ref{maxkr}) indicates how many selected nodes are repeatedly calculated $f_k$ as their direct influence.Therefore, the subtracted part expresses that any flow will only be calculated once as a direct influence,in other words, any flow will only be calculated as direct influence by only one of the selected nodes, the flow is privatized by a selected node during the selection process.
\begin{equation}
\widehat{\psi} \left( {{f_k}} \right) = \left\{ \begin{array}{l}
\left| \widehat{U} \right| - 1,{\rm{    }}\left| \widehat{U} \right| \ge 1\\
{\rm{   }}0\quad \quad \; \; ,\, \left| \widehat{U} \right| = 0
\end{array} \right.
\label{maxkr}
\end{equation}
Therefore, it is possible to iteratively calculate the top $K$ nodes with the most comprehensive influence, the highest comprehensive influence in each round will be selected, and privatize all the flows it covers (but not including the flows privatized by other selected nodes in the previous);That is, for the direct influence of the candidate $R_i$ in the $k$ round, use the flow set $\widetilde{F}^{cs}_i$ to calculate the direct influence; Let $F_k^{cs}$ denote the set of flows privatized by the selected node in the $k_{th}$ round, so $\widetilde{F}^{cs}_i = F_i^c- \bigcup_{m=1}^{k-1}F_m^{cs}$,and the subtracted portion indicates that it has been privatized by the selected node of the previous k-1 round;If $R_i$ is selected in this round, the flows in $\widetilde{F}^{cs}_i$ is $R_i$ exclusive,then $\widetilde{F}^{cs}_i$ written as $F^{cs}_k$,$R_i$ is written as $R^s_k$; In the $k+1$ round selection, even if the other candidate nodes cover these flows, these flows have been privatized by the $R_i$, will not bring value to these candidate nodes in the $k+1$ round. The K-round selection is carried out, and the nodes with the highest comprehensive influence are selected in each round,through the process of privatizing the flows with the most influential nodes in each round, any two selected nodes are: $F^{cs}_i \bigcap F^{cs}_j = \varnothing$, and the comprehensive influence of each node is the highest in the corresponding round, so the K selected nodes guarantee (\ref{maxk}) to be maximized under the constraint (\ref{maxkc}). so the $K$ nodes are the optimal solutions.$D_i^k$ indicates the direct influence of the candidate node $R_i$ in the $k-th$ round selection, which $D_i^k = |\widetilde{F}^{cs}_i|/{\left| F^c \right|}$. (\ref{maxki}) describes the iterative calculation formula for the comprehensive influence of the candidate nodes for each round of selection.
\begin{equation}
I_{i}^{k}=\alpha \cdot D_{i}^{k}+\beta \cdot {{S}_{i}}+\gamma \cdot {{H}_{i}}
\label{maxki}
\end{equation}
Algorithm 1 gives the solution process, and Fig3 also shows the change process of direct influence in each round of elections.Finally, the K node set $R^s$ and the $F^{cs}$ set corresponding to each node can be obtained.The time complexity of the algorithm is $O(K*|R|)$.


\begin{algorithm}[h]
\caption{Sampling Point Selection}
\begin{algorithmic}[1]

\STATE define $R^s=\varnothing$  //  The Set of Selected Routers

\FOR{$k=1$; $k < K$; $k++$ }

\FOR{each $R_i \in R-R^s$}
\IF{$I_i^k > max$}
\STATE $max = I_i^k$
\STATE $R^s_k = R_i$
\ENDIF
\ENDFOR
\STATE put $R^s_k$ into $R^s$
\STATE $F_K^{cs} = F_i^c- \bigcup_{m=1}^{k-1}F_m^{cs}$ 
\ENDFOR

\RETURN $R^s$
\label{code:recentEnd}
\end{algorithmic}
\end{algorithm}


 


\subsection{Non-starved influence change-aware priority polling slot allocation}

After the node is selected, the number of Slots needs to be allocated to these nodes when the constraint (10) is satisfied and the Slot sequence between nodes is not considered.The optimization model of the subproblem is the (\ref{dasd}), which is still judged by the formula (3).The goal of the model is to maximize system value under the given $R^s$ and constraints (10). The reason why the model does not need to consider the influence of overlapping flows between nodes on the total influence of the system is that we use the privatized flows set of each node $F^{cs}$ to quantify the direct influence of each node,and arbitrarily $F^{cs}_i \bigcap F^{cs}_j = \varnothing$, so the direct influence in the node $R_i$ in unit time t is (\ref{newDI}).
\begin{equation}
{D}_i=\frac{\sum_{f_k \in F^{cs}_i} P\{N_p^k(t)> 0\}}{|F^c|} 
\label{newDI}
\end{equation}
For the cost $w_i$ of the node in unit time t, still uses $F_i^c$ to calculate. This is because we only logically eliminate the overlapping flows between nodes, but there is still no change in the overall rate of the underlying nodes.$CNT[1..K]$ represents the number of slots for each node.
\begin{equation}
\max [ \alpha \cdot \delta(D_i,CNT[i]) + \frac{CNT[i]}{L}(\beta \cdot S_i + \gamma \cdot H_i)  ]
\label{dasd}
\end{equation}
In the sub-problem optimization model, the comprehensive influence of the nodes increases with the increase of the number of Slots. When the situation of ELSE in formula (3) is satisfied, another growth trend will be presented,and he trend depends on $S_i$ and $H_i$. We have described the reasons in the construction of the CIMCS model.That is to say, the value generated by a single Slot is related to the number of Slots, not independent, so it is not possible to solve the optimal solution with multiple backpacks.

We give a simple and efficient allocation algorithm to achieve: the simple influence of the node's comprehensive influence per unit time t from high to low polling allocation, so that the current high-influence nodes preferentially allocate Slot.After each round of allocation, if a node satisfies the ELSE condition of formula (3), the comprehensive influence of the node in unit time t is $\beta \cdot S_i + \cdot H_i$,which does not satisfy the formula ( 3) For ELSE condition, continue to use the comprehensive influence of the three weighted sums.The algorithm stops until $C = 0$ or $C$ cannot be allocated.Algorithm 2 gives a description of the process.The algorithm perceives the change in the comprehensive influence per t due to the change in the number of node slots.Each round guarantees that the current high-influenced nodes preferentially allocate slot, and the polling allocation method can also avoid the starvation of some low-influence nodes. $CNT[1..K]$ can be obtained.The time complexity of the algorithm is $O(\frac{C}{Min\{w_i\}})$.


\begin{algorithm}[h]
\caption{Impact Priority Polling Allocation Slots}
\begin{algorithmic}[1]
\STATE Define $CNT[1..K] = 0$;   $\widetilde{I}[1..K] = 0;$
\WHILE{$C > 0$ OR $C$ is different from the last round } 
\FOR{$each$ $R^s_i \in R^s$ }
\IF{$R_i$ satisfy$ \delta \left( {{D_i}, CNT[i] } \right) $  the $ELSE$ condition }
\STATE  $ \widetilde{I}[i]  = \frac{1}{{T}/{t}\;}\cdot (\beta \cdot {{S}_{i}}+\gamma \cdot {{H}_{i}})$ 
\ELSE 
\STATE  $ \widetilde{I}[i]  = \alpha \cdot \frac{{{v}_{i}\cdot 1}}{\left| {{F}^{c}} \right|}+\frac{1}{{T}/{t}\;} \cdot (\beta \cdot {{S}_{i}}+\gamma \cdot {{H}_{i}})$
\ENDIF
\ENDFOR
\STATE Descending sorting $ \widetilde{I}  $;
\FOR{$i=1$; $i < K$; $i++$}
\IF{$C >= w_i$}
\STATE $CNT[i]++$
\STATE $C = C - w_i $
\ENDIF
\ENDFOR
\ENDWHILE
\RETURN $CNT[1..K]$
\label{code:recentEnd}
\end{algorithmic}
\end{algorithm}


\subsection{Co-Sampling slots ordering based on effectiveness maximizing greed}
In the CIMCS model, we have described the effect of cooperative sampling between nodes on sampling accuracy and the effective ratio of sampling. As shown in Fig. (3), the cooperation between nodes is reflected in the order of sampling Slots of each node.Two nodes covering same flows should be avoided sampling in the same slot.They should be arranged with the appropriate sampling order and cooperate with each other to maximize the effectiveness of the entire sampling system during period T.For this sub-problem, formula (9) is its optimization model. The optimization goal of the model is to minimize the number of resampled flows under given the number of Slots of each node.For any two sampling points $R_i, R_j$, assuming that $S_i, S_j$ are their Slot sets respectively, then $|S_i \bigcap S_j|$ indicates the number of times they are sampled under the same Slot;$|F_i^c \bigcap F_j^c|$ indicates the number of flows that thet cover the same.Therefore, the number of flows that are resampled between two nodes can be expressed as: $|S_i \bigcap S_j| \cdot |F_i^c \bigcap F_j^c|$.How to reasonably arrange the sampling time slot sequence of each node in the period T, so that the number of resampled flows of the whole system is minimized, thereby ensuring the minimum sampling packet repetition rate, so that the effectiveness of the whole sampling system is maximized.
\begin{equation}
\min \sum{(\left| \widetilde{{{S}_{i}}}\bigcap \widetilde{{{S}_{j}}} \right|}\cdot \left| \widetilde{{{F}_{i}}}\bigcap \widetilde{{{F}_{j}}} \right|),\forall i,j\wedge j>i
\end{equation}
This problem can be solved by using the search backtracking method, but it is a problem that cannot be solved in a polynomial time. Therefore, we consider a simple greedy algorithm to solve the approximate optimal solution of the problem.Algorithm 3 gives a description of the process.The $CNT$ array has been calculated in the previous section, indicating the number of slots for each node.At the beginning of the algorithm, initialize the $M^{slot}$ two-dimensional array to store the placement relationship between $R_i$ and $s^l$: $M^{slot}[i][l]=1$, which represents $R_i$ node sampling at $s^l$.In each round, an idle slot is selected for all nodes of $\forall i, CNT[i]\neq 0$, and the selected slot makes the number of resampled flows of the whole system the least compared to other optional slots.Through each round, the nodes greedily chooses a slot that minimizes the current the number of resampled flows of the entire system, when $i, CNT[0]=0$, the slots order of each node are selected, an approximate optimal solution is obtained.The time complexity of the algorithm is $O(L^2 \cdot K)$.Fig.6 demonstrates the process.In the example, the approximate solution we solved by this algorithm is 6 and the optimal solution is 5.
\begin{algorithm}[h]
\caption{Order of Time Slot Based on Greedy}
\begin{algorithmic}[1]
\REQUIRE  $M$ ~, $S$ ~, $c_j$
\WHILE{$CNT[i] > 0,\exists i \wedge i = 1,2...,k$}
\FOR{$i=1$; $i <= K$; $i++$}
\IF{$CNT[i] > 0$}
\STATE $Min = Max$ $Integer$
\FOR{$l=1$; $l < \frac{T}{t}$; $l++$}
\IF{$M^{slot}[i][l] = 0$}
\STATE $temp = \sum^{K}_{j=1 \wedge j != i}(\left| \widetilde{{{F}_{i}}}\bigcap \widetilde{{{F}_{j}}} \right| \cdot M^{slot}[j][l]) + H[l] $
\IF{$temp < Min$}
\STATE $Min = temp$; $Sp = l$ 
\ENDIF
\ENDIF
\ENDFOR
\STATE$M^{slot}[i][Sp] = 1$; $ CNT[i]--$; $H[Sp] = Min$
\ENDIF
\ENDFOR
\ENDWHILE

\RETURN $M^{slot}$
\label{code:recentEnd}
\end{algorithmic}
\end{algorithm}

\begin{figure}[!hhhhhhhhhht]
\centering
\includegraphics[width=7cm]{images/greedy_for_order_slot.png}
\caption{Illustrating sampling node slots placement based on greedy algorithm}
\label{slot_order}
\end{figure}

\section{Experiments and results}
In order to verify the effectiveness and performance of our algorithm, we have built a laboratory bed based on floodlight controller and openvswitch + mininet. The whole experimental bed contains 12 Dell XPS hosts, 20 core CPU, and Ubuntu 16.06.2 LTS. One runs a floodlight controller that fuses our algorithm, the other runs a data collector, and the remaining 10 deploy a network topology with 110 switch nodes and 50 host nodes. The experimental traffic dataset comes from the open project "the WIDE Project". We selected data from 14:00-14:15 in August 6, 2018. After cleaning and screening, we collate 5000 data streams for experiments. In the experiment, the number of data streams changed from 1000 to 5000..
We implement four algorithms: Random-K, top-K based on the extended median centrality, top-K based on the standard median centrality, our algorithm XXX. Based on the above four algorithms, we have made a comparative experiment in three measurement mechanisms: sampling accuracy, packet repetition rate and the number of rat streams collected.
Fig.x shows the comparison of sampling accuracy in different algorithms. Our algorithm is 7\% higher than Top-k and over 20\% than the other two algorithms. From Fig.x, we can see that different algorithms do almost the same amount of elephant flow collection. In fact, our algorithm only takes more part of the rat flow than other algorithms. And Fig.x shows that our algorithm is effective in reducing duplication and reducing it by more than 30\%.

\begin{figure}[!hhhhhhhhhht]
\centering
\includegraphics[width=7cm]{images/cmp_sam_accu.png}
\caption{accuracy comparison with respect to different algorithms}
\label{aaa.png}
\end{figure}
\begin{figure}[!hhhhhhhhhht]
\centering
\includegraphics[width=7cm]{images/cmp_rep_rate.png}
\caption{repetition rate comparison with respect to different algorithms}
\label{aaa.png}
\end{figure}

%\begin{figure}[!hhhhhhhhhht]
%\centering
%\includegraphics[width=8cm]{images/cmp_rep_rate.png}
%\caption{repetition rate comparison with respect to different algorithms }
%\label{aaa.png}
%\end{figure}

%\begin{figure}[!!!!!!!!!!!!!!hhhhhhhhhht]
%\centering
%\subfigure[ comparison of sampling accuracy]
%{\includegraphics[width=4.1cm]{images/cmp_sam_accu.png}
%\label{fig_x_accu}
%}
%\subfigure[comparison of repetition rate]
%{\includegraphics[width=4.1cm]{images//cmp_rep_rate.png}
%\label{fig_x_rep}
%}
%\caption{comparison with respect to different algorithms}
%\label{fig_x_cmp}
%\end{figure}

\begin{figure}[!hhhhhhhhhht]
\centering
\includegraphics[width=8cm]{images/cmp_mice_flownum.png}
\caption{comparison of different algorithms in number of mice flow}
\label{aaa.png}
\end{figure}

\begin{figure}[!hhhhhhhhhht]
\centering
\includegraphics[width=8cm]{images/num_slot.png}
\caption{comparison of sampling flow number at different slot}
\label{aaa.png}
\end{figure}



\section{Conclusion}
\begin{itemize}
\item Lab environment
\item Sampling accuracy comparison
\item Sampling repetition rate comparison
\item Greedy centrality algorithm experimental results
 
\item Deduplication rate algorithm comparison

\item Experimental comparison of adaptive co-sampling algorithm
\end{itemize}



%\section{Conclusion}

%\begin{itemize}
%\item[-] Case 1: $ \frac{1}{SW_{num}} < S_{rate} \Leftrightarrow Confliction $   \\
%\item[-] Case 2: $ \frac{1}{SW_{num}} >= S_{rate} \Leftrightarrow no Confliction $ 


 





% trigger a \newpage just before the given reference
% number - used to balance the columns on the last page
% adjust value as needed - may need to be readjusted if
% the document is modified later
%\IEEEtriggeratref{8}
% The "triggered" command can be changed if desired:
%\IEEEtriggercmd{\enlargethispage{-5in}}

% references section

% can use a bibliography generated by BibTeX as a .bbl file
% BibTeX documentation can be easily obtained at:
% http://mirror.ctan.org/biblio/bibtex/contrib/doc/
% The IEEEtran BibTeX style support page is at:
% http://www.michaelshell.org/tex/ieeetran/bibtex/
%\bibliographystyle{IEEEtran}
% argument is your BibTeX string definitions and bibliography database(s)
%\bibliography{IEEEabrv,../bib/paper}
%
% <OR> manually copy in the resultant .bbl file
% set second argument of \begin to the number of references
% (used to reserve space for the reference number labels box)
\begin{thebibliography}{1}

\bibitem{1}
S.~Yoon, T.~Ha, S.~Kim and H.~Lim,\hskip 1em plus 0.5em minus 0.4em
Scalable Traffic Sampling using Centrality Measure on Software-Defined Networks, in \emph{IEEE Communications Magazine}, pp.43-49, July 2017.

\bibitem{1}
M.~Malboubi, L.~Wang, C.N.~Chuah, P.~Sharma,\hskip 1em plus 0.5em minus 0.4em
Intelligent SDN based Traffic (de)Aggregation and Measurement Paradigm (iSTAMP), in \emph{IEEE INFOCOM}, Apr 2014.

\bibitem{1}
L.~Tong and W.~Gao,\hskip 1em plus 0.5em minus 0.4em
Application-Aware Traffic Scheduling for Workload Offloading in Mobile Clouds, in \emph{IEEE INFOCOM}, pp.1-9, Apr 2016.

\bibitem{1}
J.~Jiang, S.~Ma, B.~Li and B.~Li,\hskip 1em plus 0.5em minus 0.4em
Symbiosis: Network-Aware Task Scheduling in Data-Parallel Frameworks, in \emph{IEEE INFOCOM}, pp.10-14, Apr 2016.

\bibitem{1}
P.~Bakopoulos, K.~Christodoulopoulos, G.~Landi et al,\hskip 1em plus 0.5em minus 0.4em
NEPHELE: An End-to-End Scalable and Dynamically Reconfgurable Optical Architecture for Application-Aware SDN Cloud Data Centers, in \emph{IEEE Communitions Magazine}, pp.178-188, Feb 2018.

\bibitem{2}
J.~Xu, J.Y.~Wang, Q.~Qi, H.F.~Sun and B.~He,\hskip 1em plus 0.5em minus 0.4em
IARA: An Intelligent Application-aware VNF for Network Resource Allocation with Deep Learning, in \emph{IEEE SECON}, pp.1-3, June 2018.

\bibitem{2}
J.~Suh, T.T.~Kwon, C.~Dixon, W.~Felter and J.~Carter,\hskip 1em plus 0.5em minus 0.4em
OpenSample: A Low-latency, Sampling-based Measurement Platform for Commodity SDN, in \emph{IEEE ICDCS}, pp.228-237, July 2014.

\bibitem{2}
Z.~Su, T.~Wang, Y.~Xia and M.~Hamdi,\hskip 1em plus 0.5em minus 0.4em
CeMon: A Cost-effective Flow Monitoring System in Software Defined Networks, in \emph{Computer Networks}, pp.101-115, Dec 2015.

\bibitem{3}
N.F.~Huang, C.C.~Li, C.H.~Li, C.C.~Chen,C.H.~C and I.H.~Hsu,\hskip 1em plus 0.5em minus 0.4em
Application Identification System or SDN QoS based on Machine Learning and DNS Responses, in \emph{APNOMS}, pp.407-410, Sept 2017.

\bibitem{3}
S.~Jeong, D.~Lee, J.~Hyun, J.~Li, and J.W.~Hong,\hskip 1em plus 0.5em minus 0.4em
Application-aware Traffic Engineering in Software-Defined Network, in \emph{APNOMS}, pp.315-318, Sept 2017.

\bibitem{3}
G.~Cheng and Y.~Tang,\hskip 1em plus 0.5em minus 0.4em
eOpenFlow: Software Defined Sampling via a Highly Adoptable OpenFlow Extension, in \emph{IEEE ICC}, pp.1-6, May 2017.


\bibitem{3}
S.~Zhao and D.~Medhi,\hskip 1em plus 0.5em minus 0.4em
Application Performance Optimization Using Application-Aware Networking, in \emph{IEEE NOMS},pp.1-6, Apr 2018.

\bibitem{4}
M.~Malboubi, S.M.~Peng, P.~Sharma and C.N.~Chuah,\hskip 1em plus 0.5em minus 0.4em
A Learning-based Measurement Framework for Traffic Matrix Inference in Software Defined Networks, in \emph{Computers \& Electrical Engineering}, Dec 2017.

\bibitem{4}
K.~Bilal, S.U.~Khan, L.~Zhang et al,\hskip 1em plus 0.5em minus 0.4em
Quantitative comparisons of the state-of-the-art data center architectures, in \emph{Concurrency \& Computation Practice \& Experience}, Dec 2017.

\end{thebibliography}




% that's all folks
\end{document}
