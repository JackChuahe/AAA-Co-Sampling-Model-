
%% bare_conf_compsoc.tex
%% V1.4b
%% 2015/08/26
%% by Michael Shell
%% See:
%% http://www.michaelshell.org/
%% for current contact information.
%%
%% This is a skeleton file demonstrating the use of IEEEtran.cls
%% (requires IEEEtran.cls version 1.8b or later) with an IEEE Computer
%% Society conference paper.
%%
%% Support sites:
%% http://www.michaelshell.org/tex/ieeetran/
%% http://www.ctan.org/pkg/ieeetran
%% and
%% http://www.ieee.org/

%%*************************************************************************
%% Legal Notice:
%% This code is offered as-is without any warranty either expressed or
%% implied; without even the implied warranty of MERCHANTABILITY or
%% FITNESS FOR A PARTICULAR PURPOSE! 
%% User assumes all risk.
%% In no event shall the IEEE or any contributor to this code be liable for
%% any damages or losses, including, but not limited to, incidental,
%% consequential, or any other damages, resulting from the use or misuse
%% of any information contained here.
%%
%% All comments are the opinions of their respective authors and are not
%% necessarily endorsed by the IEEE.
%%
%% This work is distributed under the LaTeX Project Public License (LPPL)
%% ( http://www.latex-project.org/ ) version 1.3, and may be freely used,
%% distributed and modified. A copy of the LPPL, version 1.3, is included
%% in the base LaTeX documentation of all distributions of LaTeX released
%% 2003/12/01 or later.
%% Retain all contribution notices and credits.
%% ** Modified files should be clearly indicated as such, including  **
%% ** renaming them and changing author support contact information. **
%%*************************************************************************


% *** Authors should verify (and, if needed, correct) their LaTeX system  ***
% *** with the testflow diagnostic prior to trusting their LaTeX platform ***
% *** with production work. The IEEE's font choices and paper sizes can   ***
% *** trigger bugs that do not appear when using other class files.       ***                          ***
% The testflow support page is at:
% http://www.michaelshell.org/tex/testflow/



\documentclass[conference,compsoc]{IEEEtran}
% Some/most Computer Society conferences require the compsoc mode option,
% but others may want the standard conference format.
%
% If IEEEtran.cls has not been installed into the LaTeX system files,
% manually specify the path to it like:
% \documentclass[conference,compsoc]{../sty/IEEEtran}





% Some very useful LaTeX packages include:
% (uncomment the ones you want to load)


% *** MISC UTILITY PACKAGES ***
%
%\usepackage{ifpdf}
% Heiko Oberdiek's ifpdf.sty is very useful if you need conditional
% compilation based on whether the output is pdf or dvi.
% usage:
% \ifpdf
%   % pdf code
% \else
%   % dvi code
% \fi
% The latest version of ifpdf.sty can be obtained from:
% http://www.ctan.org/pkg/ifpdf
% Also, note that IEEEtran.cls V1.7 and later provides a builtin
% \ifCLASSINFOpdf conditional that works the same way.
% When switching from latex to pdflatex and vice-versa, the compiler may
% have to be run twice to clear warning/error messages.






% *** CITATION PACKAGES ***
%
\ifCLASSOPTIONcompsoc
  % IEEE Computer Society needs nocompress option
  % requires cite.sty v4.0 or later (November 2003)
  \usepackage[nocompress]{cite}
\else
  % normal IEEE
  \usepackage{cite}
\fi

\usepackage{caption}
\usepackage{graphicx, subfigure}
\usepackage{algorithm}
\usepackage{algorithmic}
\usepackage{multirow}
\usepackage{enumitem}
\usepackage{amsmath,amssymb,amsfonts}
\renewcommand{\algorithmicrequire}{ \textbf{Input:}} %Use Input in the format of Algorithm
\renewcommand{\algorithmicensure}{ \textbf{Output:}} %UseOutput in the format of Algorithm


\renewcommand\thesection{\arabic{section}} 
\renewcommand\thesubsection{\thesection.\Alph{subsection}} 
% cite.sty was written by Donald Arseneau
% V1.6 and later of IEEEtran pre-defines the format of the cite.sty package
% \cite{} output to follow that of the IEEE. Loading the cite package will
% result in citation numbers being automatically sorted and properly
% "compressed/ranged". e.g., [1], [9], [2], [7], [5], [6] without using
% cite.sty will become [1], [2], [5]--[7], [9] using cite.sty. cite.sty's
% \cite will automatically add leading space, if needed. Use cite.sty's
% noadjust option (cite.sty V3.8 and later) if you want to turn this off
% such as if a citation ever needs to be enclosed in parenthesis.
% cite.sty is already installed on most LaTeX systems. Be sure and use
% version 5.0 (2009-03-20) and later if using hyperref.sty.
% The latest version can be obtained at:
% http://www.ctan.org/pkg/cite
% The documentation is contained in the cite.sty file itself.
%
% Note that some packages require special options to format as the Computer
% Society requires. In particular, Computer Society  papers do not use
% compressed citation ranges as is done in typical IEEE papers
% (e.g., [1]-[4]). Instead, they list every citation separately in order
% (e.g., [1], [2], [3], [4]). To get the latter we need to load the cite
% package with the nocompress option which is supported by cite.sty v4.0
% and later.





% *** GRAPHICS RELATED PACKAGES ***
%
\ifCLASSINFOpdf
  % \usepackage[pdftex]{graphicx}
  % declare the path(s) where your graphic files are
  % \graphicspath{{../pdf/}{../jpeg/}}
  % and their extensions so you won't have to specify these with
  % every instance of \includegraphics
  % \DeclareGraphicsExtensions{.pdf,.jpeg,.png}
\else
  % or other class option (dvipsone, dvipdf, if not using dvips). graphicx
  % will default to the driver specified in the system graphics.cfg if no
  % driver is specified.
  % \usepackage[dvips]{graphicx}
  % declare the path(s) where your graphic files are
  % \graphicspath{{../eps/}}
  % and their extensions so you won't have to specify these with
  % every instance of \includegraphics
  % \DeclareGraphicsExtensions{.eps}
\fi
% graphicx was written by David Carlisle and Sebastian Rahtz. It is
% required if you want graphics, photos, etc. graphicx.sty is already
% installed on most LaTeX systems. The latest version and documentation
% can be obtained at: 
% http://www.ctan.org/pkg/graphicx
% Another good source of documentation is "Using Imported Graphics in
% LaTeX2e" by Keith Reckdahl which can be found at:
% http://www.ctan.org/pkg/epslatex
%
% latex, and pdflatex in dvi mode, support graphics in encapsulated
% postscript (.eps) format. pdflatex in pdf mode supports graphics
% in .pdf, .jpeg, .png and .mps (metapost) formats. Users should ensure
% that all non-photo figures use a vector format (.eps, .pdf, .mps) and
% not a bitmapped formats (.jpeg, .png). The IEEE frowns on bitmapped formats
% which can result in "jaggedy"/blurry rendering of lines and letters as
% well as large increases in file sizes.
%
% You can find documentation about the pdfTeX application at:
% http://www.tug.org/applications/pdftex





% *** MATH PACKAGES ***
%
%\usepackage{amsmath}
% A popular package from the American Mathematical Society that provides
% many useful and powerful commands for dealing with mathematics.
%
% Note that the amsmath package sets \interdisplaylinepenalty to 10000
% thus preventing page breaks from occurring within multiline equations. Use:
%\interdisplaylinepenalty=2500
% after loading amsmath to restore such page breaks as IEEEtran.cls normally
% does. amsmath.sty is already installed on most LaTeX systems. The latest
% version and documentation can be obtained at:
% http://www.ctan.org/pkg/amsmath





% *** SPECIALIZED LIST PACKAGES ***
%
%\usepackage{algorithmic}
% algorithmic.sty was written by Peter Williams and Rogerio Brito.
% This package provides an algorithmic environment fo describing algorithms.
% You can use the algorithmic environment in-text or within a figure
% environment to provide for a floating algorithm. Do NOT use the algorithm
% floating environment provided by algorithm.sty (by the same authors) or
% algorithm2e.sty (by Christophe Fiorio) as the IEEE does not use dedicated
% algorithm float types and packages that provide these will not provide
% correct IEEE style captions. The latest version and documentation of
% algorithmic.sty can be obtained at:
% http://www.ctan.org/pkg/algorithms
% Also of interest may be the (relatively newer and more customizable)
% algorithmicx.sty package by Szasz Janos:
% http://www.ctan.org/pkg/algorithmicx




% *** ALIGNMENT PACKAGES ***
%
%\usepackage{array}
% Frank Mittelbach's and David Carlisle's array.sty patches and improves
% the standard LaTeX2e array and tabular environments to provide better
% appearance and additional user controls. As the default LaTeX2e table
% generation code is lacking to the point of almost being broken with
% respect to the quality of the end results, all users are strongly
% advised to use an enhanced (at the very least that provided by array.sty)
% set of table tools. array.sty is already installed on most systems. The
% latest version and documentation can be obtained at:
% http://www.ctan.org/pkg/array


% IEEEtran contains the IEEEeqnarray family of commands that can be used to
% generate multiline equations as well as matrices, tables, etc., of high
% quality.




% *** SUBFIGURE PACKAGES ***
%\ifCLASSOPTIONcompsoc
%  \usepackage[caption=false,font=footnotesize,labelfont=sf,textfont=sf]{subfig}
%\else
%  \usepackage[caption=false,font=footnotesize]{subfig}
%\fi
% subfig.sty, written by Steven Douglas Cochran, is the modern replacement
% for subfigure.sty, the latter of which is no longer maintained and is
% incompatible with some LaTeX packages including fixltx2e. However,
% subfig.sty requires and automatically loads Axel Sommerfeldt's caption.sty
% which will override IEEEtran.cls' handling of captions and this will result
% in non-IEEE style figure/table captions. To prevent this problem, be sure
% and invoke subfig.sty's "caption=false" package option (available since
% subfig.sty version 1.3, 2005/06/28) as this is will preserve IEEEtran.cls
% handling of captions.
% Note that the Computer Society format requires a sans serif font rather
% than the serif font used in traditional IEEE formatting and thus the need
% to invoke different subfig.sty package options depending on whether
% compsoc mode has been enabled.
%
% The latest version and documentation of subfig.sty can be obtained at:
% http://www.ctan.org/pkg/subfig




% *** FLOAT PACKAGES ***
%
%\usepackage{fixltx2e}
% fixltx2e, the successor to the earlier fix2col.sty, was written by
% Frank Mittelbach and David Carlisle. This package corrects a few problems
% in the LaTeX2e kernel, the most notable of which is that in current
% LaTeX2e releases, the ordering of single and double column floats is not
% guaranteed to be preserved. Thus, an unpatched LaTeX2e can allow a
% single column figure to be placed prior to an earlier double column
% figure.
% Be aware that LaTeX2e kernels dated 2015 and later have fixltx2e.sty's
% corrections already built into the system in which case a warning will
% be issued if an attempt is made to load fixltx2e.sty as it is no longer
% needed.
% The latest version and documentation can be found at:
% http://www.ctan.org/pkg/fixltx2e


%\usepackage{stfloats}
% stfloats.sty was written by Sigitas Tolusis. This package gives LaTeX2e
% the ability to do double column floats at the bottom of the page as well
% as the top. (e.g., "\begin{figure*}[!b]" is not normally possible in
% LaTeX2e). It also provides a command:
%\fnbelowfloat
% to enable the placement of footnotes below bottom floats (the standard
% LaTeX2e kernel puts them above bottom floats). This is an invasive package
% which rewrites many portions of the LaTeX2e float routines. It may not work
% with other packages that modify the LaTeX2e float routines. The latest
% version and documentation can be obtained at:
% http://www.ctan.org/pkg/stfloats
% Do not use the stfloats baselinefloat ability as the IEEE does not allow
% \baselineskip to stretch. Authors submitting work to the IEEE should note
% that the IEEE rarely uses double column equations and that authors should try
% to avoid such use. Do not be tempted to use the cuted.sty or midfloat.sty
% packages (also by Sigitas Tolusis) as the IEEE does not format its papers in
% such ways.
% Do not attempt to use stfloats with fixltx2e as they are incompatible.
% Instead, use Morten Hogholm'a dblfloatfix which combines the features
% of both fixltx2e and stfloats:
%
% \usepackage{dblfloatfix}
% The latest version can be found at:
% http://www.ctan.org/pkg/dblfloatfix




% *** PDF, URL AND HYPERLINK PACKAGES ***
%
%\usepackage{url}
% url.sty was written by Donald Arseneau. It provides better support for
% handling and breaking URLs. url.sty is already installed on most LaTeX
% systems. The latest version and documentation can be obtained at:
% http://www.ctan.org/pkg/url
% Basically, \url{my_url_here}.




% *** Do not adjust lengths that control margins, column widths, etc. ***
% *** Do not use packages that alter fonts (such as pslatex).         ***
% There should be no need to do such things with IEEEtran.cls V1.6 and later.
% (Unless specifically asked to do so by the journal or conference you plan
% to submit to, of course. )


% correct bad hyphenation here
\hyphenation{op-tical net-works semi-conduc-tor}



\begin{document}
%
% paper title
% Titles are generally capitalized except for words such as a, an, and, as,
% at, but, by, for, in, nor, of, on, or, the, to and up, which are usually
% not capitalized unless they are the first or last word of the title.
% Linebreaks \\ can be used within to get better formatting as desired.
% Do not put math or special symbols in the title.
\title{AAA: High Agile Adaptive Flow-Awareness \\ Network for SDN}


% author names and affiliations
% use a multiple column layout for up to three different
% affiliations
\author{He Cai$^{1}$, Jun Deng$^{1}$, Xiaofei Wang$^{1}$
\\
${^1}$Tianjin Key Laboratory of Advanced Networking, %School of Computer Science and Technology,\\
Tianjin University, Tianjin, China.
%\begin{small}
%*Prof. Keqiu Li is the corresponding author.
%\end{small}
}


% conference papers do not typically use \thanks and this command
% is locked out in conference mode. If really needed, such as for
% the acknowledgment of grants, issue a \IEEEoverridecommandlockouts
% after \documentclass

% for over three affiliations, or if they all won't fit within the width
% of the page (and note that there is less available width in this regard for
% compsoc conferences compared to traditional conferences), use this
% alternative format:
% 
%\author{\IEEEauthorblockN{Michael Shell\IEEEauthorrefmark{1},
%Homer Simpson\IEEEauthorrefmark{2},
%James Kirk\IEEEauthorrefmark{3}, 
%Montgomery Scott\IEEEauthorrefmark{3} and
%Eldon Tyrell\IEEEauthorrefmark{4}}
%\IEEEauthorblockA{\IEEEauthorrefmark{1}School of Electrical and Computer Engineering\\
%Georgia Institute of Technology,
%Atlanta, Georgia 30332--0250\\ Email: see http://www.michaelshell.org/contact.html}
%\IEEEauthorblockA{\IEEEauthorrefmark{2}Twentieth Century Fox, Springfield, USA\\
%Email: homer@thesimpsons.com}
%\IEEEauthorblockA{\IEEEauthorrefmark{3}Starfleet Academy, San Francisco, California 96678-2391\\
%Telephone: (800) 555--1212, Fax: (888) 555--1212}
%\IEEEauthorblockA{\IEEEauthorrefmark{4}Tyrell Inc., 123 Replicant Street, Los Angeles, California 90210--4321}}




% use for special paper notices
%\IEEEspecialpapernotice{(Invited Paper)}




% make the title area
\maketitle

% As a general rule, do not put math, special symbols or citations
% in the abstract
\begin{abstract}
With the proliferation of Internet applications and the explosion of traffic, Fine-gaine's flow-level information acquisition provides basic support for network management, TE, security analysis, and Qos.In traffic sampling, due to the scale of the network, the analysis capcity of the Collector,and the highly random and dynamic network environment, there are a large number of uncapable flows. In this paper, we focus on maximizing the sampling accuracy, and proposing the Influence Maximization Model(IMM) from the three dimensions of sampling node selection, time allocation and collaboration between nodes.Based on the optimization model, we propose three heuristic algorithms to solve the approximate optimal solution. We implemented the AAA platform and evaluated the performance of the algorithms using real network topology. 
\end{abstract}

% no keywords




% For peer review papers, you can put extra information on the cover
% page as needed:
% \ifCLASSOPTIONpeerreview
% \begin{center} \bfseries EDICS Category: 3-BBND \end{center}
% \fi
%
% For peerreview papers, this IEEEtran command inserts a page break and
% creates the second title. It will be ignored for other modes.
\IEEEpeerreviewmaketitle



\section{Introduction}

With the data traffic and network scale rapidly increaing, there exists huge demand for scalable network management. Meanwhile, network monitoring and application awareness play a increasingly critical role in Quality of Service(Qos),Traffic Engineering(TE) and cyber security.
Briefly, application awareness is a basic technology to enhance  automation and intelligence of the network. It is divided into two processes: packet acquisition and traffic identification. Packet acquisition refers to capturing packets from switches through a mechanism or an algorithm. Traffic identification refers to parsing the five-tuple information of packets from different layer according to OSI model ,then  recognizing the application layer protocol with the help of DPI tools. Application-aware network can improve the visibility of itself , promote integration of different business and eliminate  faults quickly .However, the application awareness need integrate the high precision,high efficiency with real time, which is still a challange owing to the volume and variety of data in the large-scale network.

Software-defined network (SDN) is a new technical architecture which decouple the network control plane from the data-forwarding plane. It advocates building an open and programmable network to provide flexible, central controlled(or centralized) and globally visible network services, through which SDN can facilitate the operation and maintenance of the data center(DC) network. In a software-defined network, packet acquisition depends on OpenFlow(OF) protocol,which is varied from the Netflow and Sflow used in traditional networks.

Based on port, payload, and traffic behavior characteristics, DPI can identify a variety of  information including the application layer protocol of a data flow, and be applied in application-aware network. In traditional networks, DPI devices are bound to the data plane, which makes it impossible to visualize global fine-grained traffic in real time. Therefore,many people are concerned about the research and optimization of the combination of SDN and application awareness. However, most of the current solutions are to deploy DPI in the SDN controller.In this case, parsing each single packet will be  computationally heavy for controller. In addition, network scale, number of sampling nodes, sampling frequency and repetition rate of packet all increase performance consumption of controllers.On the other hand, in order to improve the accuracy of application recognition, the system must be able to capture continuous  packets of the same flow regarding to the characteristics of DPI.
To solve the above problems, a agile, adaptive and cooperative sampling mechanism which can be applied to large-scale data center network is urgently needed. 


\begin{figure}[!hhhhhhhhhht]
\centering
\includegraphics[width=8.5cm]{images/png_architecture.png}
\caption{The System Architecture Of AAA}
\label{Architecture}
\end{figure}

\section{Related Work}
Our main contributions are summarized as follows:
\begin{itemize}[leftmargin=*]
\setlength{\parindent}{0pt}

\item In the context of Flow-level Sampling, we focus on maximizing the sampling accuracy. From the three dimensions of  node selection,sampling node cooperation and time allocation, the IMM impact maximization model is constructed. We first explained the effect of cooperative sampling between nodes and packet repetition rate on sampling accuracy.
\item Based on the IMM optimization model, we split it into three sub-problems: sampling node election, Slot time allocation, and cooperation between nodes, and three heuristic algorithms are proposed to solve the approximate optimal solution of IMM.
\item We built the AAA platform and evaluated the performance of the algorithms using a real large-scale WAN topology. Finally, an application identification demo is implemented: App-Awareness.
\end{itemize}

\section{SYSTEM DESCRIPTION AND PROBLEM STATEMENT }

Figure \ref{Architecture} shows the framework of AAA. In the AAA mode, the SPS mode is used to collect the flow table entries of each group. All packets representing all the flows passing through the switch are copied to the unified group table entry to perform the actions defined by the group entry. When the controller controls the initialization of the group of entries, the action is initialized to point to the collector or discarded. Therefore, when the controller needs to control a switch to sample or stop sampling, it only needs to simply send a Group Mod message to the corresponding switch. When the action is Drop, the sampling is stopped. When it is directed to the exit of the collector, the sampling is started. . This not only makes use of the pure OpenFlow protocol, but also controls it more precisely based on the controller's global. Assuming that the sampling period is $T$, through the adaptive coordination algorithm of AAA, the sampling points are selected and the sampling duration is allocated for them, and the sampling order is determined and the strategy is decentralized to the corresponding switch to make them cooperate sampling.

Sampling in large-scale networks, while satisfying the low intrusion of the network and the limitation of the Collector (IDS e.g) analysis capability, maximizing the sampling accuracy of the Flow-level and increasing the sampling efficiency ratio is a huge challenge.Therefore, highly agile and adaptive algorithms are needed to set the sampling strategy based on the real-time conditions of the current network. When the network size is large, assuming that the number of switches is $n$, $K(K<=n)$ nodes are generally selected for sampling, and $K$ is determined according to actual conditions.In each sampling period $T$, if the capacity of the collector is $C$ $packets/T$; Within $T$, the total number of sampling packets must be less than or equal to $C$.For flow-level sampling, under the given $C$ and $K$ constraints, the reasons for affecting the sampling accuracy include the selection of sampling nodes and the allocation of sampling time for each node.Another reason should be the sampling effective ratio (sampled number of non-repetitive packets/number of total sampling packets).The collection of repeated packets not only occupies a limited sampling resource, but also limits the sampling accuracy, and also reduces the effectiveness of the upper application (IDS e.g).The repeated packets is generated because multiple switching nodes cover same flows and are sampled at the same time. Therefore, \textit{each sampling node should consider the overlapping relationship of the flows, and have reasonable coordination in the respective sampling time schedules}, so that the overlapping time of the system can be as small as possible, thereby reducing the repetition rate and improving the sampling accuracy.
\subsection{IMM  Model}





The most reasonable node selection and time allocation, the optimal co-sampling strategy to maximize the flow-level sampling accuracy. Therefore, we consider from these three perspectives to model the maximum flow-level sampling accuracy. First we propose an intuitive perspective based on maximizing area coverage to analyze the problem. Figure 2 shows this idea: In a sampling period T, the set of flows F in the entire network is like an area depicted by a solid red line. R1..Rn, each node is covered by 0 or more streams, the set $F_i^c$, which is a shaded area in gray in Fig. 2, which represents the value of the node for the coverage of the known stream in the whole network. recorded as dynamic value. The red dashed area, which is the newly arrived stream that node Ri may cover in T, represents the potential value of the node in T time: these streams arrive after the sampling strategy is decentralized and are therefore not perceived by the sampling algorithm . The overlapping area between the gray areas is the overlapping part of the flow covered between the nodes, $F_i^c \bigcap F_j^c$. The overlapping portion between the red dotted areas is the overlapping portion of the newly arrived flow covered between the nodes. Some nodes contain both gray and red areas, The greater the number of streams detected by the node, the greater the value of the node. Therefore, the Flow-Level sampling problem can be intuitively converted into an area coverage maximization problem. That is, under the given collector processing power and other constraints, the sampling time allocation of each node is realized, so that the coverage area is maximized (the maximum number of stream coverage).


On the problem of maximizing area coverage, for the same stream, if there is temporal overlap in sampling on different nodes, the overlap should be subtracted when calculating the overall coverage area. However, the red region of each node is unknown to the sampling strategy. Even if the distribution model of the flow can be analyzed, the overlapping relationship of these unknown flows at each node cannot be known. Therefore the potential value of a node requires an independent (independent of other nodes) quantization. For a node, its intermediateity in the Topology [2] and the proportion of its historical flow can be used to evaluate the potential impact of the node, and these influences represent the potential value of the node, namely: potential impact The greater the force, the greater the value that the node may create in a unit of time. The value generated by the node during unit time t should be equal to the dynamic value $+$ potential value. Assume that the arrival of the packet of the stream $f_i$ obeys the Poisson distribution of $da$, so if any switch captures at least one packet of $f_i$, then the stream is considered to be successfully captured, $f_i$ The probability of being captured in unit time t is $P{N_p^k (t)>0}$. For $R_i$, if it is assigned a t, the dynamic value brought by the node (actually the number of expected to capture the known stream) is $  $, then its area coverage is $D_i/|F^c|$, the quantization method is based on an extended version of [2] The mediation center metric, which we call the dynamic impact of $R_i$ over the t-sampling duration. The OSPF-based mediation in Topology measures the influence of nodes in the topology. We use a standardized intermediate degree [2], and the influence of $R_i$ in the topology over the entire sampling period T is recorded as $S_i$. If the higher the $S_i$, the node is likely to go through more streams. If at some point in the network, the dynamic influence of $R_i$ is greater than $R_j$, that is, the current $R_i$ has more streams, but the dynamic impact of $R_j$ is higher than $R_i$, so in the following time, $R_j$ has a greater probability of going through more streams. The historical flow ratio of node $R_i$ reflects the activity of the node throughout the network life cycle. We call it the historical influence of $R_i$ over the entire sampling period T, denoted as $H_i $, $H_i = TF_i /TF$. The potential influence of $R_i$ in period T is the combination of $S_i$ and $H_i$. The combined influence of $R_i$ on the sampling time per unit time t can be quantified as $(t<=T)$, which is the weighted sum of aby.

Therefore, we present the Impact Maximization Quantization Model (IMM) formula (1), which uses the influence maximization approach to maximize the sampling of the stream. Since the time is continuous, first let t be the unit time length, where $t<=T, let l = T/t$, which means that there are 1 unit Slot in the T period, and each slot is recorded as s1...sl. Where $^S_i$ represents the set of Slots allocated for $R_i$. When $|S|$ is larger, the comprehensive influence of the nodes is larger, but (3) gives the comprehensive influence of the nodes with Slot. The number of judgment conditions is increased. When $v_i*|^S| >|F^c_i|$ is: the node realizes full capture of the current stream passing through it, then more Slot allocation will not continue to enhance its dynamic influence, and will only continue to enhance the potential impact. force. Therefore, the combined influence of all nodes is summed, and the dynamic influence of the overlapping part of the whole system is subtracted to obtain the total influence of the system.




\begin{equation}
\begin{split}
%\begin{gather}
\max \sum\limits_{i}^{n}{(\alpha \cdot \frac{\delta ({{v}_{i}},\left| \widetilde{{{S}_{i}}} \right|)}{\left| {{F}^{c}} \right|}+\frac{\left| \widetilde{{{S}_{i}}} \right|}{{T}/{t}\;}\cdot (\beta \cdot {{S}_{i}}+\gamma \cdot {{H}_{i}}))}  \\
-\frac{\alpha }{\left| {{F}^{c}} \right|}\cdot \sum\limits_{{{f}_{k}}\in {{F}^{c}}}{\sum\limits_{l=1}^{{T}/{t}\;}{\left( P\left\{ N_{p}^{k}\left( t \right)>0 \right\}\cdot \psi \left( {{f}_{k}},{{s}^{l}} \right) \right)}}
%\end{gather}
\end{split}
\end{equation}
subject to:
 
\begin{equation}
\delta \left( {{v_i},\left| {\widetilde {{S_i}}} \right|} \right) = \left\{ \begin{array}{l}
{v_i} \cdot \left| {\widetilde {{S_i}}} \right|,{\rm{    }}{v_i} \cdot \left| {\widetilde {{S_i}}} \right| < \left| {F_i^c} \right|\\
\left| {F_i^c} \right|{\rm{   }}\quad,\ {\rm{    ELSE}}
\end{array} \right.
\end{equation}
\begin{equation}
U = \left\{ {{R_i},{f_k} \in F_i^c \wedge {s^l} \in \widetilde {{S_i}}} \right\}
\end{equation}
\begin{equation}
\psi \left( {{f_k},{s^l}} \right) = \left\{ \begin{array}{l}
\left| U \right| - 1,{\rm{    }}\left| U \right| \ge 1\\
{\rm{   }}0\quad \quad \; \; ,\, \left| U \right| = 0
\end{array} \right.
\end{equation}
 
\begin{equation}
\sum\nolimits_i^n {f(\widetilde {{S_i}}) \le } K,\quad f(\widetilde {{S_i}}) = \left\{ \begin{array}{l}
1,\ \ \left| {\widetilde {{S_i}}} \right| \ge 1\\
0,\ \ \left| {\widetilde {{S_i}}} \right| = 0
\end{array} \right.
\end{equation}
\begin{equation}
\sum\nolimits_i^n {{w_i} \cdot \varphi (\widetilde {{S_i}},{s^l}) \le C \cdot } t,{\rm{  }}\forall {{\rm{s}}^l} \wedge \varphi (\widetilde {{S_i}},{s^l}) = \left\{ \begin{array}{l}
0,{s^l} \notin \widetilde {{S_i}}\\
t,{s^l} \in \widetilde {{S_i}}
\end{array} \right.
\end{equation}
 
 

\begin{figure}[!!!!!!!!!!!!!!hhhhhhhhhht]
\centering
\subfigure[area coverage]
{\includegraphics[width=8cm]{images/area_coverage.png}
\label{fig_1_area}
}

\subfigure[time slot allocation]
{\includegraphics[width=8.5cm]{images/slot_num_order.png}
\label{fig_1_slot}
}
 
\caption{overview of model}
\label{fig_1_model}
\end{figure}


 

For the model of maximizing impact problems, the coverage of known traffic, the coverage of unknown new flows, the selection of nodes, the allocation of time slots, and the scheduling of Slot time series are fully considered. In the scheduling of time Slot, because the overlapping flows between nodes will cause conflicts under the same Slot, the overall coverage is reduced, so in order to get the optimal solution, the more overlapping the two nodes, if they are both When more than 0 Slots are allocated, the number of Slots they overlap should be as small as possible, which in turn solves another important sampling problem: how to reduce the problem of duplicate packets. Because each node contains the same stream, in order to optimize the results, it will be mutually exclusive on the Slot. Therefore, in the optimized solution, the sequence of $\widetilde S_i$ of each node can make the area covered by overlapping sampling in the whole cycle. Minimized, thus greatly reducing the repetition rate of the package while ensuring maximum sampling accuracy. Therefore, the nodes are cooperative, and through the constraint relationship between each other, the flow coverage relationship, the static topology relationship, the activity degree, etc., and finally maximize the number of sampled streams. (In the sampling process, when multiple nodes are sampled, a large number of duplicate packets are generated, because the stream may pass through any number of nodes, and the same packet will be collected at multiple nodes, which not only wastes valuable sampling resources, At the same time, the sampling accuracy is reduced. The repeated packets are reduced as much as possible, and the problem is avoided by the superposition relationship between the nodes, which not only improves the sampling accuracy, but also improves the efficiency of the upper application of the collector).

The model contains a number of sub-constrained sub-problems. We consider the complexity and feasibility while quantifying, and it is actually difficult to directly solve the solution to the problem. Therefore, we decompose the problem model into three sub-models: node selection, Slot number allocation, and Slot sequence arrangement. For each sub-problem, we model it independently and use an independent algorithm to solve the optimal solution or approximate optimal solution. In the end, the approximation of the problem can be effectively obtained.
This problem is not only a variant of multiple backpacks, but also adds enough constraints on the issue of multiple backpacks. Asking to solve this problem is an NP-hard problem. Therefore, we break it down into three parts to approximate the problem.


\section{Optimal}
In this section, we decompose the IMM into three sub-problems: K sampling point Selection, Slot time allocation, node cooperative sampling optimization, and corresponding three heuristic algorithms are proposed to form the approximate optimal solution of the IMM model.
\subsection{$K$ Sampling Point Selection} 
In the selection of K sampling points, we only focus how to use the static properties of the nodes to quantify their comprehensive influence, which is independent of time allocation. Therefore, we do not care about the effect of the number of slots in (1) on the comprehensive influence of the nodes (next section will show you how to make a reasonable allocation of slot number with selected points).The optimization model for this subproblem is equation (7), where $D_i=$.The $max{dsdada}$ part of equation (7) expresses that any flow will only have a direct influence value for one of all nodes, and its direct influence on other nodes will be subtracted.In other words, any flow is privatized by a node during sampling nodes selection.Therefore, it is possible to iteratively calculate the top $K$ nodes with the most comprehensive influence, the highest comprehensive influence in each round will be selected, and privatize all the flows it covers (but not including the flows privatized by other selected nodes in the previous selection);That is, for the direct influence value of the candidate $R_i$ in the $k$ round, the flow set included is $F^{cs}_i = F_i- \bigcup_{m=1}^{k-1}F_m^c$;If $R_i$ is selected in this round,  the flows in $F^{cs}_i$ is $R_i$ exclusive;In the $k+1$ round selection, even if the candidate nodes cover these flows, these flows have been privatized by the $R_i$, will not bring value to these candidate nodes in the $k+1$ round. The K-round selection is carried out, and the nodes with the highest comprehensive influence are selected in each round, so that the sum of the comprehensive influences of the K nodes is guaranteed to be maximized, so the K nodes are the optimal solutions.$D_i^k$ indicates the direct influence of the candidate node $R_i$ in the k-th round selection, which is calculated as equation (8). Equation (11) describes the iterative calculation formula for the comprehensive influence of the candidate nodes for each round of selection.Algorithm 1 gives the solution process, and Fig3 also shows the change process of direct influence in each round of elections.Finally, the K node set $R^s$ and the $F{cs}$ set corresponding to each node can be obtained.

\begin{equation}
D_{i}^{k}={\left| {{F}_{i}}-\bigcup\nolimits_{c}^{k-1}{\widetilde{{{F}_{c}}}} \right|}/{\left| F-\bigcup\nolimits_{c}^{k-1}{\widetilde{{{F}_{c}}}} \right|}\;
\end{equation}
 

\begin{equation}
I_{i}^{k}=\alpha \cdot D_{i}^{k}+\beta \cdot {{S}_{i}}+\gamma \cdot {{H}_{i}}
\end{equation}

\begin{algorithm}[h]
\caption{Sampling Point Selection}
\begin{algorithmic}[1]
\REQUIRE ~~\\ The set of routers: $ R$ \\  The size of node will be selected: $K$ \\ The current flow information matrix: $M$
\STATE define $R^s=\{\}$  //  The Set of Selected Routers

\FOR{$k=1$; $k < K$; $k++$ }

\FOR{each $R_i \in R-R^s$}
\IF{$I_i^k > max$}
\STATE $max = I_i^k$
\STATE $SR = R_i$
\ENDIF
\ENDFOR
\STATE put $SR$ to $R^s$
\STATE mark $SR$ as $R_k$ in $R^s$
\ENDFOR

\RETURN $R^s$
\label{code:recentEnd}
\end{algorithmic}
\end{algorithm}


 





\subsection{Allocation of Time Slot}

After the node is selected, the number of Slots needs to be allocated to these nodes when the constraint (10) is satisfied and the Slot sequence between nodes is not considered.For each sample node, each value is assigned to a certain value, we still use the comprehensive influence to quantify the value of the node.The optimization model of the sub-problem is the formula (8), and the direct influence judgment of the formula is still the formula (3).The optimization model of the subproblem is the formula (8), which is still judged by the formula (3).And use the flow set privatized by the node to calculate the direct influence of the node in the unit time t $D_i$, $D_i=\frac{\sum_{f_k \in F^{cs}_i} P\{N_p^k(t)> 0\}}{|F^c|} $.For the cost $w_i$ of the node in unit time t, the dynamic influence $D_i$ in equation (7) still uses $F_i^c$ to calculate not $F^{cs}_i$. This is because we only logically eliminate the overlapping flows between nodes, but there is still no change in the overall rate of the underlying nodes.In the sub-problem optimization model, the comprehensive influence of the nodes increases with the increase of the number of Slots. When the situation of ELSE in formula (3) is satisfied, another growth trend will be presented,and he trend depends on $S_i$ and $H_i$. We have described the reasons in the construction of the IMM model.That is to say, the value generated by a single Slot is related to the number of Slots, not independent, so it is not possible to solve the optimal solution with multiple backpacks.We give a simple and efficient allocation algorithm to achieve: the simple influence of the node's comprehensive influence per unit time t from high to low polling allocation, so that the current high-influence nodes preferentially allocate Slot.After each round of allocation, if a node satisfies the ELSE condition of formula (3), the comprehensive influence of the node in unit time t is $\beta \cdot S_i + \cdot H_i$,which does not satisfy the formula ( 3) For ELSE condition, continue to use the comprehensive influence of the three weighted sums.The algorithm stops until $C = 0$ or $C$ cannot be allocated.Algorithm 2 gives a description of the process.The algorithm perceives the change in the comprehensive influence per t due to the change in the number of node slots.Each round guarantees that the current high-influenced nodes preferentially allocate slot, and the polling allocation method can also avoid the starvation of some low-influence nodes.
\begin{algorithm}[h]
\caption{Impact Priority Polling Allocation Slots}
\begin{algorithmic}[1]
\STATE Define $CNT[1..K] = 0$;   $\widetilde{I}[1..K] = 0;$
\WHILE{$C > 0$ OR $C$ is different from the last round } 
\FOR{$each$ $R^s_i \in R^s$ }
\IF{$R_i$ satisfy$ \delta \left( {{v_i},\left| CNT[i] \right|} \right) $  the $ELSE$ condition }
\STATE  $ \widetilde{I}[i]  = \frac{1}{{T}/{t}\;}\cdot (\beta \cdot {{S}_{i}}+\gamma \cdot {{H}_{i}})$ 
\ELSE 
\STATE  $ \widetilde{I}[i]  = \alpha \cdot \frac{{{v}_{i}\cdot 1}}{\left| {{F}^{c}} \right|}+\frac{1}{{T}/{t}\;} \cdot (\beta \cdot {{S}_{i}}+\gamma \cdot {{H}_{i}})$
\ENDIF
\ENDFOR
\STATE Descending sorting $ \widetilde{I}  $;
\FOR{$i=1$; $i < K$; $i++$}
\IF{$C >= w_i$}
\STATE $CNT[i]++$
\STATE $C = C - w_i $
\ENDIF
\ENDFOR
\ENDWHILE
\RETURN $CNT$
\label{code:recentEnd}
\end{algorithmic}
\end{algorithm}


\subsection{Order of Time Slot}
In the IMM model, we describe the effect of cooperative sampling between nodes on sampling accuracy and the effective ratio of sampling. As shown in Fig. (3), the cooperation between nodes is reflected in the order of sampling Slots of each node.For this sub-problem, formula (9) is its optimization model. The optimization goal of the model is to minimize the number of resampled flows under given the number of Slots of each node.For any two sampling points $R_i, R_j$, assuming that $S_i, S_j$ are their Slot sets respectively, then $|S_i \bigcap S_j|$ indicates the number of times they are sampled under the same Slot;$|F_i^c \bigcap F_j^c|$ indicates the number of flows that thet cover the same.Therefore, the number of flows that are resampled between two nodes can be expressed as: $|S_i \bigcap S_j| \cdot |F_i^c \bigcap F_j^c|$.How to reasonably arrange the sampling time slot sequence of each node in the period T, so that the number of resampled flows of the whole system is minimized, thereby ensuring the minimum sampling packet repetition rate, so that the effectiveness of the whole sampling system is maximized.
\begin{equation}
\min \sum{(\left| \widetilde{{{S}_{i}}}\bigcap \widetilde{{{S}_{j}}} \right|}\cdot \left| \widetilde{{{F}_{i}}}\bigcap \widetilde{{{F}_{j}}} \right|),\forall i,j\wedge i\ne j
\end{equation}

This problem can be solved by using the search backtracking method, but it is a problem that cannot be solved in a polynomial time. Therefore, we consider a simple greedy algorithm to solve the approximate optimal solution of the problem.Algorithm 3 gives a description of the process.The $CNT$ array has been calculated in the previous section, indicating the number of slots for each node.At the beginning of the algorithm, initialize the $M^{slot}$ two-dimensional array to store the placement relationship between $R_i$ and $s^l$: $M^slot[i][l] = 1$, which represents $R_i$ node is sampled at $s^l$.In each round, an idle slot is selected for all nodes of $i, CNT[i]!=0$, and the selected slot makes the number of resampled flows of the whole system the least compared to other optional slots.Through each round, the nodes greedily chooses a slot that minimizes the current the number of resampled flows of the entire system, when $i, CNT[0]=0$, the slots order of each node is selected, an approximate optimal solution is obtained.Fig.6 demonstrates the process. In the example, the approximate solution we solved by this algorithm is 6 and the optimal solution is 5.
\begin{algorithm}[h]
\caption{Order of Time Slot Based on Greedy}
\begin{algorithmic}[1]
\REQUIRE  $M$ ~, $S$ ~, $c_j$
\WHILE{$CNT[i] > 0,\exists i \wedge i = 1,2...,k$}
\FOR{$i=1$; $i <= K$; $i++$}
\IF{$CNT[i] > 0$}
\STATE $Min = Max$ $Integer$
\FOR{$l=1$; $l < \frac{T}{t}$; $l++$}
\IF{$M^{slot}[i][l] = 0$}
\STATE $temp = \sum^{K}_{j=1 \wedge j != i}(\left| \widetilde{{{F}_{i}}}\bigcap \widetilde{{{F}_{j}}} \right| \cdot M^{slot}[j][l]) + H[l] $
\IF{$temp < Min$}
\STATE $Min = temp$; $Sp = l$ 
\ENDIF
\ENDIF
\ENDFOR
\STATE$M^{slot}[i][Sp] = 1$; $ CNT[i]--$; $H[Sp] = Min$
\ENDIF
\ENDFOR
\ENDWHILE

\RETURN $M^{slot}$
\label{code:recentEnd}
\end{algorithmic}
\end{algorithm}

\begin{figure}[!hhhhhhhhhht]
\centering
\includegraphics[width=8cm]{images/greedy_for_order_slot.png}
\caption{Illustrating sampling node slots placement based on greedy algorithm}
\label{slot_order}
\end{figure}

\section{Experiments and results}
In order to verify the effectiveness and performance of our algorithm, we have built a laboratory bed based on floodlight controller and openvswitch + mininet. The whole experimental bed contains 12 Dell XPS hosts, 20 core CPU, and Ubuntu 16.06.2 LTS. One runs a floodlight controller that fuses our algorithm, the other runs a data collector, and the remaining 10 deploy a network topology with 110 switch nodes and 50 host nodes. The experimental traffic dataset comes from the open project "the WIDE Project". We selected data from 14:00-14:15 in August 6, 2018. After cleaning and screening, we collate 5000 data streams for experiments. In the experiment, the number of data streams changed from 1000 to 5000..
We implement four algorithms: Random-K, top-K based on the extended median centrality, top-K based on the standard median centrality, our algorithm XXX. Based on the above four algorithms, we have made a comparative experiment in three measurement mechanisms: sampling accuracy, packet repetition rate and the number of rat streams collected.
Fig.x shows the comparison of sampling accuracy in different algorithms. Our algorithm is 7\% higher than Top-k and over 20\% than the other two algorithms. From Fig.x, we can see that different algorithms do almost the same amount of elephant flow collection. In fact, our algorithm only takes more part of the rat flow than other algorithms. And Fig.x shows that our algorithm is effective in reducing duplication and reducing it by more than 30\%.

%\begin{figure}[!hhhhhhhhhht]
%\centering
%\includegraphics[width=8cm]{images/cmp_sam_accu.png}
%\caption{accuracy comparison with respect to different algorithms}
%\label{aaa.png}
%\end{figure}

\begin{figure}[!!!!!!!!!!!!!!hhhhhhhhhht]
\centering
\subfigure[ comparison of sampling accuracy]
{\includegraphics[width=4.1cm]{images/cmp_sam_accu.png}
\label{fig_x_accu}
}
\subfigure[comparison of repetition rate]
{\includegraphics[width=4.1cm]{images//cmp_rep_rate.png}
\label{fig_x_rep}
}
\caption{comparison with respect to different algorithms}
\label{fig_x_cmp}
\end{figure}

\begin{figure}[!hhhhhhhhhht]
\centering
\includegraphics[width=8cm]{images/cmp_mice_flownum.png}
\caption{comparison of different algorithms in number of mice flow}
\label{aaa.png}
\end{figure}

\begin{figure}[!hhhhhhhhhht]
\centering
\includegraphics[width=8cm]{images/num_slot.png}
\caption{comparison of sampling flow number at different slot}
\label{aaa.png}
\end{figure}

%\begin{figure}[!hhhhhhhhhht]
%\centering
%\includegraphics[width=8cm]{images/cmp_rep_rate.png}
%\caption{repetition rate comparison with respect to different algorithms }
%\label{aaa.png}
%\end{figure}


 





 


\section{Conclusion}
\begin{itemize}
\item Lab environment
\item Sampling accuracy comparison
\item Sampling repetition rate comparison
\item Greedy centrality algorithm experimental results
 
\item Deduplication rate algorithm comparison

\item Experimental comparison of adaptive co-sampling algorithm
\end{itemize}



%\section{Conclusion}

%\begin{itemize}
%\item[-] Case 1: $ \frac{1}{SW_{num}} < S_{rate} \Leftrightarrow Confliction $   \\
%\item[-] Case 2: $ \frac{1}{SW_{num}} >= S_{rate} \Leftrightarrow no Confliction $ 


 





% trigger a \newpage just before the given reference
% number - used to balance the columns on the last page
% adjust value as needed - may need to be readjusted if
% the document is modified later
%\IEEEtriggeratref{8}
% The "triggered" command can be changed if desired:
%\IEEEtriggercmd{\enlargethispage{-5in}}

% references section

% can use a bibliography generated by BibTeX as a .bbl file
% BibTeX documentation can be easily obtained at:
% http://mirror.ctan.org/biblio/bibtex/contrib/doc/
% The IEEEtran BibTeX style support page is at:
% http://www.michaelshell.org/tex/ieeetran/bibtex/
%\bibliographystyle{IEEEtran}
% argument is your BibTeX string definitions and bibliography database(s)
%\bibliography{IEEEabrv,../bib/paper}
%
% <OR> manually copy in the resultant .bbl file
% set second argument of \begin to the number of references
% (used to reserve space for the reference number labels box)
\begin{thebibliography}{1}

\bibitem{1}
S.~Yoon, T.~Ha, S.~Kim and H.~Lim,\hskip 1em plus 0.5em minus 0.4em
Scalable Traffic Sampling using Centrality Measure on Software-Defined Networks, in \emph{IEEE Communications Magazine}, pp.43-49, July 2017.

\bibitem{1}
M.~Malboubi, L.~Wang, C.N.~Chuah, P.~Sharma,\hskip 1em plus 0.5em minus 0.4em
Intelligent SDN based Traffic (de)Aggregation and Measurement Paradigm (iSTAMP), in \emph{IEEE INFOCOM}, Apr 2014.

\bibitem{1}
L.~Tong and W.~Gao,\hskip 1em plus 0.5em minus 0.4em
Application-Aware Traffic Scheduling for Workload Offloading in Mobile Clouds, in \emph{IEEE INFOCOM}, pp.1-9, Apr 2016.

\bibitem{1}
J.~Jiang, S.~Ma, B.~Li and B.~Li,\hskip 1em plus 0.5em minus 0.4em
Symbiosis: Network-Aware Task Scheduling in Data-Parallel Frameworks, in \emph{IEEE INFOCOM}, pp.10-14, Apr 2016.

\bibitem{1}
P.~Bakopoulos, K.~Christodoulopoulos, G.~Landi et al,\hskip 1em plus 0.5em minus 0.4em
NEPHELE: An End-to-End Scalable and Dynamically Reconfgurable Optical Architecture for Application-Aware SDN Cloud Data Centers, in \emph{IEEE Communitions Magazine}, pp.178-188, Feb 2018.

\bibitem{2}
J.~Xu, J.Y.~Wang, Q.~Qi, H.F.~Sun and B.~He,\hskip 1em plus 0.5em minus 0.4em
IARA: An Intelligent Application-aware VNF for Network Resource Allocation with Deep Learning, in \emph{IEEE SECON}, pp.1-3, June 2018.

\bibitem{2}
J.~Suh, T.T.~Kwon, C.~Dixon, W.~Felter and J.~Carter,\hskip 1em plus 0.5em minus 0.4em
OpenSample: A Low-latency, Sampling-based Measurement Platform for Commodity SDN, in \emph{IEEE ICDCS}, pp.228-237, July 2014.

\bibitem{2}
Z.~Su, T.~Wang, Y.~Xia and M.~Hamdi,\hskip 1em plus 0.5em minus 0.4em
CeMon: A Cost-effective Flow Monitoring System in Software Defined Networks, in \emph{Computer Networks}, pp.101-115, Dec 2015.

\bibitem{3}
N.F.~Huang, C.C.~Li, C.H.~Li, C.C.~Chen,C.H.~C and I.H.~Hsu,\hskip 1em plus 0.5em minus 0.4em
Application Identification System or SDN QoS based on Machine Learning and DNS Responses, in \emph{APNOMS}, pp.407-410, Sept 2017.

\bibitem{3}
S.~Jeong, D.~Lee, J.~Hyun, J.~Li, and J.W.~Hong,\hskip 1em plus 0.5em minus 0.4em
Application-aware Traffic Engineering in Software-Defined Network, in \emph{APNOMS}, pp.315-318, Sept 2017.

\bibitem{3}
G.~Cheng and Y.~Tang,\hskip 1em plus 0.5em minus 0.4em
eOpenFlow: Software Defined Sampling via a Highly Adoptable OpenFlow Extension, in \emph{IEEE ICC}, pp.1-6, May 2017.


\bibitem{3}
S.~Zhao and D.~Medhi,\hskip 1em plus 0.5em minus 0.4em
Application Performance Optimization Using Application-Aware Networking, in \emph{IEEE NOMS},pp.1-6, Apr 2018.

\bibitem{4}
M.~Malboubi, S.M.~Peng, P.~Sharma and C.N.~Chuah,\hskip 1em plus 0.5em minus 0.4em
A Learning-based Measurement Framework for Traffic Matrix Inference in Software Defined Networks, in \emph{Computers \& Electrical Engineering}, Dec 2017.

\bibitem{4}
K.~Bilal, S.U.~Khan, L.~Zhang et al,\hskip 1em plus 0.5em minus 0.4em
Quantitative comparisons of the state-of-the-art data center architectures, in \emph{Concurrency \& Computation Practice \& Experience}, Dec 2017.

\end{thebibliography}




% that's all folks
\end{document}


